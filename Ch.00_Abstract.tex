% \begin{abstract}{finnish}

% Tämä dokumentti on tarkoitettu Helsingin yliopiston tietojenkäsittelytieteen osaston opin\-näyt\-teiden ja harjoitustöiden ulkoasun ohjeeksi ja mallipohjaksi. Ohje soveltuu kanditutkielmiin, ohjelmistotuotantoprojekteihin, seminaareihin ja maisterintutkielmiin. Tämän ohjeen lisäksi on seurattava niitä ohjeita, jotka opastavat valitsemaan kuhunkin osioon tieteellisesti kiinnostavaa, syvällisesti pohdittua sisältöä.


% Työn aihe luokitellaan  
% ACM Computing Classification System (CCS) mukaisesti, 
% ks.\ \url{https://dl.acm.org/ccs}. 
% Käytä muutamaa termipolkua (1--3), jotka alkavat juuritermistä ja joissa polun tarkentuvat luokat erotetaan toisistaan oikealle osoittavalla nuolella.

% \end{abstract}

\begin{otherlanguage}{english}
\begin{abstract}

Traditionally machine learning models are build to take use of single modality. Medical diagnosis is rarely based on single source of information. Multimodal machine learning  introduces techniques like data fusion that allow models to process and use multiple modalities. This thesis examines these models in a context of medical diagnosing. 

Multiple recent literature reviews suggest potential improvements in accuracy with multimodal machine learning compared to the models utilizing single modality in the same task. However, most of the research in this area is found employing retrospective data and lack prospective validation. 

The thesis follows a structure that aims to first provide overview of the basics in machine and deep learning. Then it moves to examine data fusion and medical studies employing multimodal machine learning.  

\end{abstract}
\end{otherlanguage}

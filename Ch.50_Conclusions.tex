\chapter{Conclusions\label{conclusions}}

In this thesis, we have identified some of the fundamental concepts related to multimodal machine learning and deep learning. We have clarified the differences and identified the common approaches to data fusion. Furthermore, the challenges related to data fusion are presented with a focus on medical diagnosis. Recent studies show the utilization of different data fusion approaches combined with varying machine learning methods. In general, the research questions set for this thesis are answered, as we successfully described the basic problems and methods related to the large and still evolving field of multimodal machine learning in medicine.

Multimodal machine learning has shown potential in medical diagnosing tasks. While many problems remain unsolved, potential benefits have been demonstrated in many retrospective original studies and comprehensive reviews. Research on data fusion and multimodal aspects is expected to continue, and prospective interdisciplinary studies, that allow comparison between similar approaches are needed to validate current findings. Synthetic Data and Explainable Artificial Intelligence could provide important aspects for development and research in this field. In the medical context, research on unimodal models is also important as models that improve on a single modality can then be utilized in MML architectures. Recent advances in deep learning methods such as Transformer architecture might also provide additional tools for data fusion. 


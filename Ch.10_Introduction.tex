\chapter{Introduction\label{intro}}

Diagnosing diseases or conditions is a typical task for physicians working in healthcare. Physicians use relevant data that describe the medical condition of the patient. The data needed depends heavily on the context but can be simplified as using the current and historical data available to identify the disease or condition. Additional data can be gathered from medical imaging or lab tests (\cite{NAP21794}). The complexity of the diagnosing process varies case by case but can be seen as a classification problem in the end. 

Machine learning has been used across different industries to solve classification problems, and deep learning has shown potential for complex classification tasks. The capability of the machine learning model is highly dependent on the data that it is trained with. Deep learning has been successfully used for image segmentation and classification in radiology (\cite{lee2017deep}). In medicine, data with multiple modalities namely electronic health records (EHR) and medical images are used together to gain a better understanding of the patient during the diagnostic process. However, machine learning models traditionally expect data from a single modality. Multimodal machine learning (MML) addresses the issue by introducing data fusion for multiple different sources of modalities during the training process (\cite{8269806}).


This thesis provides a basic view of the concepts in multimodal data fusion. This is achieved by introducing basic machine and deep learning concepts and then moving to data fusion. These concepts are then reviewed through recent studies utilizing the multimodal approach in medical contexts. This thesis tries to identify the potential benefits and challenges that can be found from using multiple modalities compared to single-modality approaches. 

\bigskip

\textbf{Research Questions:}

\begin{enumerate}
    \item What are the basic principles of multimodal machine learning and data fusion?
    \item How do recent studies employing multimodal machine learning techniques in medical contexts demonstrate the potential advantages and challenges compared to single-modality approaches?
\end{enumerate}